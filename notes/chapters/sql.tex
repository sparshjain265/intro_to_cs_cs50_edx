\chapter{Database}
\section{csv files}
\begin{code}
	\inputminted{python}{codes/src7/favorites0.py}
	\caption{Read a csv file in python}
\end{code}
\clearpage
\begin{code}
	\inputminted{python}{codes/src7/favorites1.py}
	\caption{Use a dictionary to count in python}
\end{code}
\begin{code}
	\inputminted{python}{codes/src7/favorites2.py}
	\caption{Print sorted dictionary by 'keys' in python}
\end{code}
\begin{code}
	\inputminted{python}{codes/src7/favorites3.py}
	\caption{Print sorted dictionary by 'values' in python}
\end{code}
\begin{code}
	\inputminted{python}{codes/src7/favorites4.py}
	\caption{lambda function in python}
\end{code}

\section{SQL}
\subsection{Example}
Open as \mintinline{bash}{sqlite3 <dbname>}:
\begin{code}
	\begin{minted}{sql}
.mode csv
.import <filename> <tablename>
	\end{minted}
	\caption{load a csv to a db in sqlite3}
\end{code}
Now we can ask the same kind of questions:
\begin{code}
	\begin{minted}{sql}
SELECT title FROM favorites;
SELECT title FROM favorites ORDER BY title;
SELECT title, COUNT(title) FROM favorites GROUP BY title;
SELECT title, COUNT(title) FROM favorites GROUP BY title LIMIT 10;
SELECT title, COUNT(title) AS n FROM favorites GROUP BY title LIMIT 10;
SELECT title, COUNT(title) AS n FROM favorites GROUP BY title ORDER BY n DESC LIMIT 10;
	\end{minted}
	\caption{SQL querries in sqlite3}
\end{code}

\subsection{Relational Database}
With any form of data, there are four fundamental operations:
\begin{enumerate}
	\item[C:] Create
	\item[R:] Read
	\item[U:] Update
	\item[D:] Delete
\end{enumerate}
\emph{Structured Query Language} is just another programming language
mainly used for databases, has keywords attached to these:
\begin{enumerate}
	\item INSERT
	\item SELECT
	\item UPDATE
	\item DELETE \\
	      \dots
\end{enumerate}

\subsection{Syntax}
\subsubsection{Datatypes:}
\begin{enumerate}
	\item BLOB - Binary Large Object
	\item INTEGER
	      \begin{enumerate}
		      \item smallint
		      \item integer
		      \item bigint
	      \end{enumerate}
	\item NUMERIC
	      \begin{enumerate}
		      \item boolean
		      \item date
		      \item datetime
		      \item numeric(scale,precision)
		      \item time
		      \item timestamp
	      \end{enumerate}
	\item REAL
	      \begin{enumerate}
		      \item real
		      \item double precision
	      \end{enumerate}
	\item TEXT
	      \begin{enumerate}
		      \item char(n)
		      \item varchar(n)
		      \item text
	      \end{enumerate}
\end{enumerate}
\subsubsection{Functions}
\begin{enumerate}
	\item AVG
	\item COUNT
	\item DISTINCT
	\item MAX
	\item MIN\\
	      \dots
\end{enumerate}
\subsubsection{Features}
\begin{enumerate}
	\item WHERE
	\item LIKE
	\item LIMIT
	\item GROUP BY
	\item ORDER BY
	\item JOIN\\
	      \dots
\end{enumerate}
\begin{code}
	\begin{minted}{sql}
CREATE TABLE table (column type, |\dots|);
INSERT INTO table (column, |\dots|) VALUES (value, |\dots|);
SELECT columns FROM table;
SELECT title FROM favorites WHERE title LIKE "%office%";
SELECT COUNT(title) FROM favorites WHERE title LIKE "%office%";
SELECT columns FROM table WHERE condition;
UPDATE table SET column=value WHERE condition;
DELETE FROM table WHERE condition;
	\end{minted}
	\caption{SQL Syntax}
\end{code}

\clearpage
\subsection{Huge Database}
Design decisions really gonna matter. Download "title.basic.tsv.gz" for example.
\subsubsection{Fields}
\begin{enumerate}
	\item tcost : tt4786824
	\item tytleType : tvSeries
	\item primaryTitle : The Crown
	\item startYear : 2016
	\item genres : Drama, History
\end{enumerate}

\begin{code}
	\inputminted{python}{codes/src7/import2.py}
	\caption{filtering the database in python}
\end{code}

\begin{code}
	\inputminted{python}{codes/src7/search.py}
	\caption{searching the database in python}
\end{code}

\clearpage
\begin{code}
	\inputminted{python}{codes/src7/import3.py}
	\caption{using SQL in python}
\end{code}

\begin{code}
	\inputminted{python}{codes/src7/import4.py}
	\caption{import to multiple tables in SQL using python}
\end{code}

\begin{code}
	\begin{minted}{sql}
SELECT * FROM shows WHERE id IN (SELECT show_id FROM genres WHERE genre = "Comedy") AND year = 2019;
	\end{minted}
	\caption{query with multiple tables in SQL}
\end{code}

\begin{code}
	\begin{minted}{sql}
CREATE INDEX person_index ON stars (person_id);
	\end{minted}
	\caption{indexing in sql}
\end{code}

\section{Problems}
\subsection{Race Conditions}
Solution? \emph{Transactions}

\subsection{SQL Injection Attacks}
Solution? \emph{Sanitize your inputs}















