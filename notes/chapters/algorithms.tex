\chapter{Algorithms}
% 00:00:00 - Introduction
% 00:01:22 - Weeks 2 Recap
% 00:02:46 - Algorithms Demo
% 00:04:05 - Algorithms Demo (Linear Search)
% 00:06:14 - Algorithms Demo (Binary Search)
% 00:08:34 - Linear Search
\section{Linear Search}
\begin{code}
	\begin{minted}[linenos]{c}
		for i from 0 to n-1
			if ith element is 50
				return true;
		return false;
	\end{minted}
	\caption{Linear Search Pseudocode}
\end{code}
% 00:09:44 - Binary Search
\section{Binary Search}
\begin{code}
	\begin{minted}[linenos]{c}
		if no items 
			return false;
		if middle item is 50
			return true;
		else if 50 < middle item 
			search left half
		else if 50 > middle item
			search right half
	\end{minted}
	\caption{Binary Search Pseudocode}
\end{code}
% 00:11:02 - Efficiency
% 00:14:03 - Big O and Omega Notation
\section{Efficiency}
\subsection{\texorpdfstring{$\bigO$ Notation:}{}} Worst case scenario
\begin{align*}
	n^2        & : \bigO(n^2)                       \\
	n\log_n{n} & : \bigO(n\log{n})                  \\
	n          & : \bigO(n) \ (Linear Search)       \\
	n/2        & : \bigO(n)                         \\
	\log_2{n}  & : \bigO(\log{n}) \ (Binary Search) \\
	{constant} & : \bigO(1)                         \\
\end{align*}
\subsection{\texorpdfstring{$\Omega$ Notation:}{}} Best case scenario
\begin{align*}
	 & \Omega(n^2)      \\
	 & \Omega(n\log{n}) \\
	 & \Omega(n)        \\
	 & \Omega(n)        \\
	 & \Omega(\log{n})  \\
	 & \Omega(1)        \\
\end{align*}

Q: Better to have a really good $\bigO$ value or a really good $\Omega$ value?\\

A: $\bigO$, or even \emph{average} case.

\clearpage
\section{Examples}
\subsection{Linear Search}
\subsubsection{Numbers}
% 00:18:39 - numbers.c
\begin{code}
	\inputminted[linenos]{c}{codes/src3/numbers.c}
	\caption{Linear Search on numbers}
\end{code}
% \clearpage
\subsubsection{Names}
% 00:21:11 - names.c
\begin{code}
	\inputminted[linenos]{c}{codes/src3/names0.c}
	\caption{Linear Search on names}
\end{code}
% 00:26:14 - phonebook.c
% \clearpage
\subsection{Bad Design}
Correct/Working code but bad design!
\begin{code}
	\inputminted[linenos,breaklines]{c}{codes/src3/phonebook0.c}
	\caption{Linear Search in a phonebook}
\end{code}
% \clearpage
% 00:31:40 - typedef (phonebook.c)
\subsection{\texorpdfstring{Good Design - \mintinline{c}{typedef struct}}{}}
Using \mintinline{c}{typedef struct} for better design!
\begin{code}
	\inputminted[linenos,breaklines]{c}{codes/src3/phonebook1.c}
	\caption{Linear Search in phonebook with \mintinline{c}{typedef struct}}
\end{code}

% 00:37:21 - Sorting Demo
% 00:43:45 - Bubble Sort
\section{Bubble Sort}
\begin{code}
	\begin{minted}[linenos]{c}
repeat n-1 times
	for i = 0 to n-2
		if ith and i+1th elements out of order
			swap them
	\end{minted}
\end{code}
\paragraph{$\bigO(n^2)$}
\paragraph{$\Omega(n^2)$}
% 00:48:08 - Selection Sort
\section{Selection Sort}
\begin{code}
	\begin{minted}[linenos]{c}
for i from 0 to n-1
	find smallest item between ith item and last item
	swap smallest item and ith item
	\end{minted}
\end{code}
\paragraph{$\bigO(n^2)$}
\paragraph{$\Omega(n^2)$}
% 00:56:45 - Algorithm Running Times
\clearpage
\section{Better Bubble Sort}
\begin{code}
	\begin{minted}[linenos]{c}
repeat until swap
	for i = 0 to n-2
		if ith and i+1th elements out of order
			swap them
	\end{minted}
\end{code}
\paragraph{$\bigO(n^2)$}
\paragraph{$\Omega(n)$}
% 01:01:56 - Elections
% 01:03:38 - Recursion
\section{Recursion}
\begin{code}
	\begin{minted}[linenos]{c}
Pick up phone book
Open to middle of phone book
Look at page
if Smith is on page
	Call Mike
else if Smith is earlier in book
	Open to middle of left half of book
	Go back to line 3
else if Smith is later in book
	Open to middle of right half of book
	Go back to line 3
else 
	Quit
	\end{minted}
	\caption{Iteration Pseudocode}
\end{code}

Can we do a better design?

\begin{code}
	\begin{minted}[linenos]{c}
Pick up phone book
Open to middle of phone book
Look at page
if Smith is on page
	Call Mike
else if Smith is earlier in book
	Search left half of book
else if Smith is later in book
	Search right half of book
else 
	Quit
	\end{minted}
	\caption{Recursion Pseudocode}
\end{code}

% 01:06:13 - iteration.c
\begin{code}
	\inputminted[linenos,breaklines]{c}{codes/src3/iteration.c}
	\caption{Iteration C code}
\end{code}
% 01:10:28 - recursion.c
\begin{code}
	\inputminted[linenos,breaklines]{c}{codes/src3/recursion.c}
	\caption{Recursion C code}
\end{code}

% 01:17:10 - Merge Sort
\clearpage
\section{Merge Sort}
\begin{code}
	\begin{minted}[linenos]{c}
if only 1 item
	return
else
	sort left half of items
	sort right half of items
	merge sorted halves
	\end{minted}
	\caption{Merge Sort Pseudocode}
\end{code}
\paragraph{$\bigO(n\log{n})$}
\paragraph{$\Omega(n\log{n})$}
% 01:26:18 - Theta Notation
\subsection{\texorpdfstring{$\Theta$ Notation}{}}
When $\bigO = \Omega$!
% 01:27:06 - Random (Visualization)