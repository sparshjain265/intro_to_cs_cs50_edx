\chapter{JavaScript}
A programming language to make webpages more interactive!

\section{Syntax}
A lot like C.

\begin{code}
	\begin{minted}{js}
let counter = 0;
counter = counter + 1;
counter += 1;
counter++;
if (x < y)
{

}
else if (x > y)
{

}
else
{

}
while (true)
{

}
for (let i = 0; i < 50; i++)
{

}
function cough(n)
{

}
	\end{minted}
	\caption{JavaScript syntax}
\end{code}

\section{Document Object Model}
Webpage as a DOM object!

\begin{code}
	\inputminted{html}{codes/web/js/alert2.html}
	\caption{Alert using JavaScript}
\end{code}

\begin{code}
	\inputminted{html}{codes/web/js/hello.html}
	\caption{Updating webpage using JavaScript}
\end{code}

\begin{code}
	\inputminted{html}{codes/web/js/counter.html}
	\caption{Variables in a webpage using JavaScript}
\end{code}

\clearpage
\begin{code}
	\inputminted{html}{codes/web/js/background.html}
	\caption{Changing background using JavaScript}
\end{code}

\clearpage
\begin{code}
	\inputminted{html}{codes/web/js/size.html}
	\caption{Updating font size using JavaScript}
\end{code}

\clearpage
\begin{code}
	\inputminted{html}{codes/web/js/blink0.html}
	\caption{Blinking a content using JavaScript}
\end{code}

\clearpage
\begin{code}
	\inputminted{js}{codes/web/js/blink1.js}
	\caption{JavaScript code in a separate file}
\end{code}

\begin{code}
	\inputminted{html}{codes/web/js/blink1.html}
	\caption{HTML using external JavaScript file}
\end{code}

\clearpage
\begin{code}
	\inputminted[breakbefore=(.]{html}{codes/web/js/geolocation.html}
	\caption{Getting location of the user via JavaScript}
\end{code}













