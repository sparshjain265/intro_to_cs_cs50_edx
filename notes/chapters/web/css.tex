\chapter{CSS}
\paragraph{Cascading Style Sheets:} To \emph{style} webpages.

\begin{code}
	\inputminted{html}{codes/web/css/css0.html}
	\caption{inline styling in html}
\end{code}

\clearpage
\begin{code}
	\inputminted{html}{codes/web/css/css1.html}
	\caption{multiple styles within an html element}
\end{code}

\begin{code}
	\inputminted{html}{codes/web/css/css2.html}
	\caption{css classes in html}
\end{code}

\begin{code}
	\inputminted{html}{codes/web/css/css3.html}
	\caption{multiple css classes in an html element}
\end{code}

\clearpage
\begin{code}
	\inputminted{css}{codes/web/css/css3.css}
	\caption{separate css file}
\end{code}

\begin{remark}
	To link your css file in your html, do so in your \emph{head} section
	via \mintinline{html}{<link rel="stylesheet" href="styles.css">}.
\end{remark}
\begin{remark}
	We can also link multiple different css files.
\end{remark}

\begin{code}
	\inputminted{html}{codes/web/css/table.html}
	\caption{styled table in html}
\end{code}

\clearpage
\paragraph{Libraries:} Many predefined css libraries available to use.
\begin{remark}
	\emph{Bootstrap} is a popular css library.
\end{remark}

\begin{code}
	\inputminted{html}{codes/web/css/bootstrap.html}
	\caption{using bootstrap css library}
\end{code}