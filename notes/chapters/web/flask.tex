\chapter{Flask}
Python based framework to write our own web-server.

\section{Hello World}

\begin{remark}
	It is conventional to name the file `application.py'.
\end{remark}

\begin{code}
	\inputminted{python}{codes/web/flask/hello0/application.py}
	\caption{Hello World in Flask}
	\label{helloWorldFlask}
\end{code}

In Program~\ref{helloWorldFlask} variable \mintinline{python}{app} represents
the web application we will run, and \mintinline{python}{__name__} represents
the current file/program. For every route (path), we define a function to return
the content we want while visiting that route.

To run your web application, when you are in the directory, you can type
\mintinline{bash}{flash run} to run the web application.

\begin{remark}
	You might want to set the following environment variables first:
	\begin{enumerate}
		\item \mintinline{bash}{export FLASK_APP=application.py}
		\item \mintinline{bash}{export FLASK_ENV=development}
	\end{enumerate}
\end{remark}

\begin{remark}
	You can return any HTML that you want!
\end{remark}

\section{Templates}
Use templates to use external HTML files in Flask!

\begin{code}
	\inputminted{python}{codes/web/flask/hello1/application.py}
	\caption{Templates in Flask}
\end{code}

\section{Variables}
\subsection{String}
\begin{code}
	\inputminted{python}{codes/web/flask/hello2/application.py}
	\caption{Variables in Flask}
\end{code}
\begin{code}
	\inputminted{html}{codes/web/flask/hello2/templates/index.html}
	\caption{Jinja syntax for (flask) variables in HTML}
\end{code}

\subsection{Random Numbers}
\begin{code}
	\inputminted{python}{codes/web/flask/random/application.py}
	\caption{Passing Random Numbers from Flask}
\end{code}
\clearpage
\begin{code}
	\inputminted{html}{codes/web/flask/random/templates/index.html}
	\caption{Displaying random numbers in HTML}
\end{code}

\section{Conditions}
\subsection{Coin Flip}
\begin{code}
	\inputminted{python}{codes/web/flask/coin/application.py}
	\caption{Coin Flipping in flask}
\end{code}
\clearpage
\begin{code}
	\inputminted{html}{codes/web/flask/coin/templates/index.html}
\end{code}

\section{Interactive Webpage}
\subsection{Forms}
\begin{code}
	\inputminted{python}{codes/web/flask/hello3/application.py}
	\caption{Requesting arguments in flask}
\end{code}
\clearpage
\begin{code}
	\inputminted{html}{codes/web/flask/hello3/templates/index.html}
	\caption{Requesting name in HTML}
\end{code}
\begin{code}
	\inputminted{html}{codes/web/flask/hello3/templates/hello.html}
	\caption{Hello (name) in HTML}
\end{code}
\clearpage
\begin{code}
	\inputminted{html}{codes/web/flask/hello3/templates/failure.html}
	\caption{Failure page in HTML}
\end{code}

\clearpage
\section{Layouts}
Use layouts to include common HTML code.
\begin{code}
	\inputminted{html}{codes/web/flask/hello4/templates/layout.html}
	\caption{Layout HTML}
\end{code}
\begin{code}
	\inputminted{html}{codes/web/flask/hello4/templates/index.html}
	\caption{Requesting name in HTML that extends layout}
\end{code}
\begin{code}
	\inputminted{html}{codes/web/flask/hello4/templates/hello.html}
	\caption{Displaying name in HTML that extends layout}
\end{code}
\begin{code}
	\inputminted{html}{codes/web/flask/hello4/templates/failure.html}
	\caption{Failure message in HTML that extends layout}
\end{code}

\section{Tasks Application}
\begin{code}
	\inputminted{python}{codes/web/flask/tasks/application.py}
	\caption{Tasks Application using Flask}
\end{code}
\clearpage
\begin{code}
	\inputminted{html}{codes/web/flask/tasks/templates/layout.html}
	\caption{Layout for the tasks application}
\end{code}
\begin{code}
	\inputminted{html}{codes/web/flask/tasks/templates/tasks.html}
	\caption{Default page for the tasks application}
\end{code}
\clearpage
\begin{code}
	\inputminted{html}{codes/web/flask/tasks/templates/add.html}
	\caption{Add page for the tasks application}
\end{code}