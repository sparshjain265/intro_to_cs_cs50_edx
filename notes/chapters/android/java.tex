\chapter{Java}
\section{Introduction}
Use Android Studio (IDE) to build android apps. Convention for package name is
your domain name in reverse followed by app name. Use androidx (newer version
android libraries). Also need to create AVD (Android Virtual Device) to simulate
an android device to run your app. Use Java to code.

\section{Data Types}
\begin{itemize}
	\item \mintinline{java}{boolean}
	\item \mintinline{java}{double}, \mintinline{java}{float}
	\item \mintinline{java}{char}
	\item \mintinline{java}{int}
	\item \mintinline{java}{List}
	\item \mintinline{java}{Map}
	\item \mintinline{java}{String}
	\item \dots
\end{itemize}

\section{Examples}
\begin{code}
	\begin{minted}{java}
		String title = "CS50";
		int count = 50;
		count += 5;

		String title = "iOS";
		if (title.equals("iOS")) {
			System.out.println("Good Choice");
		}
		else {
			System.out.println("Maybe Next Time");
		}

		int[] values = new int[]{1, 2, 3};
		for (int i = 0; i < values.length; i++){
			System.out.println(i);
		}
	\end{minted}
	\caption{First few lines of java}
\end{code}

\section{Generics}
\subsection{Lists}
\begin{code}
	\begin{minted}{java}
		List<String> values = new ArrayList<>();
		values.add("one");
		values.add("two");
		for (String value : values){
			System.out.println(value);
		}
	\end{minted}
	\caption{Lists in java using Generics}
\end{code}

\subsection{Maps}
\begin{code}
	\begin{minted}{java}
		Map<String, String> airports = new HashMap<>();
		airports.put("SFO", "San Francisco");
		airports.put("BOS", "Boston");
		for (Map.Entry<String, String> e : airports.entrySet()) {
			System.out.println(e.getKey() + ": " + e.getValue());
		}
	\end{minted}
	\caption{Maps in java using Generics}
\end{code}

\section{Classes}
structs + functions/methods = class

\begin{code}
	\begin{minted}{java}
		public class Person {
			String name;

			Person(String name) {
				this.name = name;
			}

			public void sayHello() {
				System.out.println("I'm " + name);
			}
		}

		Person person = new Person("Tommy");
		person.sayHello();
	\end{minted}
	\caption{Classes in java}
\end{code}

\section{Static Methods}
Can be called from a class, without having an instance of it.

\begin{code}
	\begin{minted}{java}
		public class Person {
			|\dots|
			public static void wave() {
				System.out.println("Wave");
			}
		}

		Person.wave();
	\end{minted}
	\caption{Static Methods in java}
\end{code}

\clearpage
\section{Inheritance}
\begin{code}
	\begin{minted}{java}
		public class Vehicle {
			public int wheels() {
				return 4;
			}

			public void go() {
				System.out.println("zoom!");
			}
		}

		public class Motorcycle extends Vehicle {
			@Override
			public int wheels() {
				return 2;
			}
		}
	\end{minted}
	\caption{Inheritance in Java Classes}
\end{code}

\section{Interfaces}
Basically a list of methods to implement in classes. If we forget, compiler raises
an error.
\begin{code}
	\begin{minted}{java}
		public interface Teacher() {
			public void teach();
		}

		public class CS50Teacher implements Teacher {
			@Override
			public void teach() {
				|\dots|
			}
		}
	\end{minted}
	\caption{Interfaces in Java Classes}
\end{code}

\begin{remark}
	We can implement multiple interfaces but only extend one class.
\end{remark}

\section{Packages}
Sort of a way to organise java code.
\begin{code}
	\begin{minted}{java}
		package edu.harvard.cs50.example;

		import java.util.List; 
	\end{minted}
	\caption{Packages in Java}
\end{code}

\section{Android}
\begin{code}
	\inputminted{java}{codes/android/javaexample/app/src/main/java/com/example/javaexample/House.java}
	\caption{House class in Java}
\end{code}
\pagebreak
\begin{code}
	\inputminted{java}{codes/android/javaexample/app/src/main/java/com/example/javaexample/MainActivity.java}
	\caption{Example Android Application in Java}
\end{code}