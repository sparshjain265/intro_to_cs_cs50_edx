\chapter{UI}
\section{Gradle}
Open Source Project - an Android Build System!

\section{MVC}
Model - View - Controller.

Design Pattern that advocates separating out the app in three different
pieces.

\section{Activities}
Sort of a base class for a screen.

\section{Resources}
All the stuff that isn't code!
Example? Layout!

\section{Layouts}
A layout describes how a view should look.
Defined using XML.

\section{XML}
eXtensible Markup Language.
\begin{code}
	\begin{minted}{xml}
		<LinearLayout>
		<TextView android:text="Hello" />
		</LinearLayout>
	\end{minted}
	\caption{sample XML code}
\end{code}

\section{Intent}
Special object that represents a way to go from one activity to another.

\section{Recycler View}
It basically represents anything that's a list of items.

\section{App Files}
From our \emph{Pokedex} App.
\subsection{Manifests}
\subsubsection{AndroidManifest.xml}
Basically a configuration file.
\begin{code}
	\inputminted{xml}{codes/android/Pokedex/app/src/main/AndroidManifest.xml}
	\caption{App Files : Manifests : AndroidManifest.xml}
\end{code}
\subsection{Java Codes}
Code like we've seen before.
\subsection{Layouts}
\begin{code}
	\inputminted{xml}{codes/android/Pokedex/app/src/main/res/layout/activity_main.xml}
	\caption{App Files : Layouts : activity\_main.xml}
\end{code}
\subsection{Values}
\subsubsection{strings.xml}
Helps when we want to support multiple languages.
\begin{code}
	\inputminted{xml}{codes/android/Pokedex/app/src/main/res/values/strings.xml}
	\caption{App Files : Values : strings.xml}
\end{code}

\subsection{Gradle Scripts}
\begin{code}
	\inputminted{groovy}{codes/android/Pokedex/app/build.gradle}
	\caption{App Files : Gradle Scripts : build.gradle (Module app)}
\end{code}

\section{Adding Recycler View}
\begin{enumerate}
	\item Add dependency in build.grade (Module: App)
	\item Start with view - What the app needs to be doing
	      \begin{enumerate}
		      \item Add view in the layout (activity\_main.xml)
		      \item Add ID to the view to reference
		      \item Need a way to define how each row is going to look
		            like
		      \item Create new layout for that
		      \item Add view in this layout, and IDs to reference
	      \end{enumerate}
	\item Then create models to power that view
	      \begin{enumerate}
		      \item Create java class to represent a single element
		      \item Add constructors, getters, and setters as per need
	      \end{enumerate}
	\item Write the controllers to hook up the two
	      \begin{enumerate}
		      \item Recycler class has another class attached to it
		            called the adapter - what data is to be displayed
		            and how to do it
		      \item Create class to represent all of the data inside
		            the recycler view that extends
		            \mintinline{java}{RecyclerView.Adapter}
		      \item It's a generic class that takes as its type a
		            \emph{ViewHolder} that holds a view and allows to
		            manipulate what's on the screen. We're going to create
		            an object that holds that view and from there we
		            modify some of the layout elements we just defined
		      \item Add fields in the ViewHolder class to represent
		            the layout and views we created
		      \item Write constructors to get views by id
		      \item Get data (or hardcode some for now)
		      \item Implement methods defined on RecyclerView.Adapter
		            \begin{enumerate}
			            \item onCreateViewHolder
			            \item onBindViewHolder
			            \item getItemCount
		            \end{enumerate}
	      \end{enumerate}
	\item Use the adapter
	      \begin{enumerate}
		      \item Add a few more fields in the MainActivity
		            \begin{enumerate}
			            \item RecyclerView
			            \item RecyclerView.Adapter
			            \item RecyclerView.LayoutManager
		            \end{enumerate}
		      \item Instantiate them
		      \item Connect them
	      \end{enumerate}
\end{enumerate}

\section{Adding New Activity}
\begin{enumerate}
	\item Create New Activity (right click on left hand side)
	\item Start with layout
	\item Next is model
	\item Now \emph{Intent}, that is how we pass data from first activity to the second
	      \begin{enumerate}
		      \item Use containerView.setTag and pass the object representing the data
		      \item Add eventHandler
	      \end{enumerate}
\end{enumerate}