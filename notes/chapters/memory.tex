\chapter{Memory}
Removing the training wheels \mintinline{c}{#include <cs50.h>} from now!

\section{Hexadecimal}
\paragraph{Digits:}$\set{1, 2, 3, 4, 5, 6, 7, 8, 9, A, B, C, D, E, F}$
\paragraph{Ambiguity:} Prefix the number with $0x$

\section{Addresses}
\begin{code}
	\inputminted[linenos, breaklines]{c}{codes/src4/address0.c}
	\caption{integer}
\end{code}

\clearpage
\begin{code}
	\inputminted[linenos, breaklines]{c}{codes/src4/address1.c}
	\caption{address of an integer}
\end{code}

\begin{code}
	\inputminted[linenos, breaklines]{c}{codes/src4/address2.c}
	\caption{address2.c}
\end{code}

\subsection{Operators}
\begin{align*}
	\text{\mintinline{c}{    & }}                         & = \text{Get the address} \\
	\text{\mintinline{c}{*}} & = \text{Go to the address}                            \\
\end{align*}

\clearpage
\section{Pointers}
\begin{code}
	\inputminted[linenos, breaklines]{c}{codes/src4/address3.c}
	\caption{accessing an address}
\end{code}

\begin{code}
	\inputminted[linenos, breaklines]{c}{codes/src4/address4.c}
	\caption{pointers}
\end{code}

\clearpage
\section{Strings}
There are no strings. Strings are just pointers.
\begin{code}
	\inputminted[linenos, breaklines]{c}{codes/src4/address5.c}
	\caption{strings}
\end{code}
\begin{code}
	\inputminted[linenos, breaklines]{c}{codes/src4/address6.c}
	\caption{strings are pointers}
\end{code}
\begin{code}
	\inputminted[linenos, breaklines]{c}{codes/src4/address7.c}
	\caption{strings are \mintinline{c}{char[]} \\
		addresses are consecutive in arrays}
\end{code}
\begin{code}
	\inputminted[linenos, breaklines]{c}{codes/src4/address8.c}
	\caption{accessing characters in a string}
\end{code}
\begin{code}
	\inputminted[linenos, breaklines]{c}{codes/src4/address10.c}
	\caption{accessing characters in a \mintinline{c}{char *}}
\end{code}

\section{String Comparision}
\begin{code}
	\inputminted[linenos, breaklines]{c}{codes/src4/compare0.c}
	\caption{comparing integers}
\end{code}
\begin{code}
	\inputminted[linenos, breaklines]{c}{codes/src4/compare1.c}
	\caption{attempting to compare strings directly}
\end{code}
\begin{code}
	\inputminted[linenos, breaklines]{c}{codes/src4/compare4.c}
	\caption{comparing strings properly}
\end{code}

\clearpage
\section{String Copy}
\begin{code}
	\inputminted[linenos, breaklines]{c}{codes/src4/copy0.c}
	\caption{attempting to copying strings directly}
\end{code}
\begin{code}
	\inputminted[linenos, breaklines]{c}{codes/src4/copy1.c}
	\caption{copy strings properly}
\end{code}

Just use \mintinline{c}{strcpy(target, source)} to copy strings.

\section{Malloc and Free}
\paragraph{\mintinline{c}{malloc}:} Allocate Memory and return its address.
\paragraph{\mintinline{c}{free}:} Free Memory (prevent leaking).

\section{Buffer Overflow}
\begin{code}
	\inputminted[linenos, breaklines, breakanywhere]{c}{codes/src4/memory.c}
	\caption{buffer overflow}
\end{code}

\section{Swap}
Pass by \emph{value} vs pass by \emph{reference}
\begin{code}
	\inputminted[linenos, breaklines]{c}{codes/src4/noswap.c}
	\caption{naive attempt at swap}
\end{code}
\clearpage
\begin{code}
	\inputminted[linenos, breaklines]{c}{codes/src4/swap.c}
	\caption{swap}
\end{code}

\clearpage
\section{\texorpdfstring{\mintinline{c}{scanf}}{}}
\begin{code}
	\inputminted[linenos, breaklines]{c}{codes/src4/scanf0.c}
	\caption{scanning an integer}
\end{code}
\begin{code}
	\inputminted[linenos, breaklines]{c}{codes/src4/scanf1.c}
	\caption{scanning a string in unintialized}
\end{code}
\clearpage
\begin{code}
	\inputminted[linenos, breaklines]{c}{codes/src4/scanf2.c}
	\caption{scanning a long string in small array}
\end{code}

\section{File I/O}
\begin{code}
	\inputminted[linenos, breaklines]{c}{codes/src4/phonebook.c}
	\caption{files in c}
\end{code}
\begin{code}
	\inputminted[linenos, breaklines]{text}{codes/src4/phonebook.csv}
	\caption{phonebook.csv}
\end{code}
\begin{code}
	\inputminted[linenos, breaklines]{c}{codes/src4/jpeg.c}
	\caption{check jpeg or not}
\end{code}




